\documentclass{article}

\title{Report on the Reference Solution to the A4 CompSys Assignment}
\author{Troels Henriksen (athas@sigkill.dk)}
\date{\today}

\begin{document}

\maketitle

\section{Introduction}

This report documents the reference solution of the A4 assignment.  It
is intended not just to explain the reference solution, but also as an
example of a report with good structure and content.  You may use it
as inspiration for reports you write later during this and other
courses.  However, do not follow it slavishly: different problems call
for different solutions, and different people have different style.
If you have another style you prefer, then stick to it.  An
idiosyncratic report executed well is superior to a conventional
report executed poorly.

The problem to be solved is to finish the implementation of a C
library of stream transducers.  The library API was defined in the
handed-out header files, and I have not found reason to stray from
it.  However, Section~\ref{sec:limitations} notes some deficiencies
that arise from limitations in the API itself.

\section{Implementation}

Due to the requirement for multiple stream transducers (and sources)
to execute concurrently, I have chosen to implement both as
subprocesses.  The process structure of a program using the
transducers API will thus contain a single main process, with one
child process for every active source or transducer.  Sinks are
executed in the main process itself, as specified in the assignment
text.

I have chosen to implement a stream as a pointer to a \texttt{FILE}
object, along with a flag indicating whether the stream already has an
associated reader (either a transducer or sink):

\begin{verbatim}
struct stream {
  FILE *f;
  int linked;
};
\end{verbatim}

The purpose of the flag is to signal an error if we try to read the
same stream from multiple transducers, as required by the assignment
text.  Memory is allocated for the \texttt{stream} object whenever
streams are created, and freed when \texttt{transducers\_free\_stream()}
is called.  This is the only place where dynamic allocation is used.

Most of the functions for creating transducers work in approximately
the same way, and I will not include the code here.  The general
approach for creating a transducer, given the input stream(s), is as
follows:

\begin{enumerate}
\item Create a pipe object (or two, in the case of
  \texttt{transducers\_dup()}).
\item Fork a new child process.
\item In the child, close the \textit{read} end of the pipe.
\item In the parent (the main process), close the \textit{write} end
  of the pipe.
\item Produce a \texttt{stream} object containing the \textit{read}
  end of the pipe.
\end{enumerate}

This has the effect that the only process that can write to the pipe
is the child, which means that when the child terminates, we may get
an end-of-file condition when trying to read from the pipe.  We can
then write transducers (and sinks) that loop until end-of-file is
reached on the input stream.

\section{Testing}

Apart from the handed-out test programs, several more were written
that test various other properties of the API, including error
detection.  All tests can be run with \texttt{make test}.  The
properties tested are:

\begin{tabular}{lp{8cm}}
  \textbf{Program} & \textbf{Property} \\\hline

  \texttt{test0.c} & The functions \texttt{transducers\_link\_source()} and \texttt{transducers\_link\_sink()} can be connected and reproduces the output of the source function. \\

  \texttt{test1.c} & A simple single-input transducer works. \\

  \texttt{test1.c} & A simple two-input transducer works. \\

  \texttt{test1.c} & \texttt{transducers\_dup()} works. \\

  \texttt{test\_error0.c} & A sink cannot be connected to a stream that has already been connected. \\

  \texttt{test\_error1.c} & A single-input transducer cannot be connected to a stream that has already been connected. \\

  \texttt{test\_error2.c} & A two-input transducer cannot be connected to a stream that has already been connected. \\
\end{tabular}

\subsection{Test Limitations}

The developed tests do not contain transducers that produce more or
fewer output elements than they receive as input.  The
\texttt{divisible.c} demo program does contain such a transducer, but
it is not a test program per se, as it is not checked whether its
output valid.

We also do not have tests that check the absence of memory leaks in
the library.  Instead, this has been verified manually via
\texttt{valgrind}.

\section{Limitations and Potential Problems}
\label{sec:limitations}

The transducers library as implemented has some issues:

\begin{itemize}
\item An error causing a transducer to crash may be silently ignored.
  Since the termination of a transducer process will implicitly cause
  its streams to be closed (by the kernel), this is not
  distinguishable from a transducer terminating normally.  The
  transducers API contains no mechanism for propagating out-of-band
  diagnostic information.
\item The child processes implementing the transducers are not reaped
  by the main process.  This may cause zombies to pile up, unless the
  main process manually calls \texttt{wait()} an appropriate number of
  times.  Conceptually, all the child processes should be reaped once
  a transducer network finishes, but this is not possible in the
  current implementation, as the child PIDs are not stored anywhere,
  and the API design does leave us any obvious place to store them.  A
  global variable would be inappropriate, as it would mean only a
  single transducer network could be constructed at a time.
\end{itemize}

\section{Conclusions}

I have implemented every part of the transducers library as specified
in the assignment text.  I judge the implementatio to be of adequate
quality.  The tests verify the correct operation of all API functions,
as well as the salient error cases.  The main limitations in the
implementation are caused by the specified API itself, which we are
unfortunately required to implement as given.

\end{document}

%%% Local Variables:
%%% mode: latex
%%% TeX-master: t
%%% End:
